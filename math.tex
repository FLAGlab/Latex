% !TEX root = keeping-up.tex

%%
\section{Math}
\label{sec:math}

\everymath{\displaystyle}

There are multiple environments to define math equations. The must common environment is to use 
\$ \$ as in-line math formulas that you can use, common to list sub indexes $x_1, x_2, \ldots , x_n$.

%%%%
\subsection{Equations}

Equations are defined by the \spy{equation} environment. Equations are numbered (if you do not want numbers, use * at the end of the word. This is a common convention in \LaTeX), and can be referenced as \fref{eq:quadratic}

\begin{equation} \label{eq:quadratic}
  x = \frac{-b \pm \sqrt{b^2 - 4ac}}{2a}
\end{equation}

Unnumbered equations can also be created using the quick equation environment, generating a math mode between \spy{\[} and \spy{\]}.

\[
Q_t(a) := \frac{\sum\limits^{t-1}_{i=1} r_t \mathbbm{1}_{a_t=a}}{\sum\limits_{i=1}^{t-1} \mathbbm{1}_{a_t=a}} 
\]

\[
\mathbbm{1}_{a_t} = \begin{cases}
       1 & \quad \text{si se escoge } a \\
       0 & \quad \text{si no se escoge } a \\
     \end{cases}
\]

%%%%
\subsection{Alignment}

\begin{align*}
f(u) & =\sum\limits_{j=1}^{n^2+1} x_jf(u_j)&\\
     & =\sum_{j=1}^{n} x_j \sum_{i=1}^{m} a_{ij}v_i&\\
     & =\sum_{j=1}^{n} \sum_{i=1}^{m} a_{ij}x_jv_i
\end{align*}

\begin{flalign}
f(u) & =\sum\limits_{j=1}^{n^2+1} x_jf(u_j)&\\
     & =\sum_{j=1}^{n} x_j \sum_{i=1}^{m} a_{ij}v_i&\\
     & =\sum_{j=1}^{n} \sum_{i=1}^{m} a_{ij}x_jv_i
\end{flalign}


\endinput