% !TEX root = keeping-up.tex

%%
\section{Math}
\label{sec:math}

\everymath{\displaystyle}

There are multiple environments to define math equations. The must common environment is to use 
\$ \$ as in-line math formulas that you can use, common to list sub indexes $x_1, x_2, \ldots , x_n$.

%%%%
\subsection{Equations}

Equations are defined by the \spy{equation} environment. Equations are numbered (if you do not want numbers, use * at the end of the word. This is a common convention in \LaTeX), and can be referenced as \fref{eq:quadratic}

\begin{equation} \label{eq:quadratic}
  x = \frac{-b \pm \sqrt{b^2 - 4ac}}{2a}
\end{equation}

Unnumbered equations can also be created using the quick equation environment, generating a math mode between \spy{\[} and \spy{\]}.

\[
Q_t(a) := \frac{\sum\limits^{t-1}_{i=1} r_t \mathbbm{1}_{a_t=a}}{\sum\limits_{i=1}^{t-1} \mathbbm{1}_{a_t=a}} 
\]

\[
\mathbbm{1}_{a_t} = \begin{cases}
       1 & \quad \text{si se escoge } a \\
       0 & \quad \text{si no se escoge } a \\
     \end{cases}
\]

%%%%
\subsection{Alignment}

\begin{align*}
f(u) & =\sum\limits_{j=1}^{n^2+1} x_jf(u_j)&\\
     & =\sum_{j=1}^{n} x_j \sum_{i=1}^{m} a_{ij}v_i&\\
     & =\sum_{j=1}^{n} \sum_{i=1}^{m} a_{ij}x_jv_i
\end{align*}

\begin{flalign}
f(u) & =\sum\limits_{j=1}^{n^2+1} x_jf(u_j)&\\
     & =\sum_{j=1}^{n} x_j \sum_{i=1}^{m} a_{ij}v_i&\\
     & =\sum_{j=1}^{n} \sum_{i=1}^{m} a_{ij}x_jv_i
\end{flalign}

%%%%
\subsection{Fancy equations}

\vspace{1cm}

\begin{equation*} \label{eq:QL}
\scriptstyle
    Q(s_{t+1},\, a_{t+1}) \leftarrow  
    {\tikzmarknode{qt}{\highlight{purple}{$Q(s_t, a_t)$}}} +
    {\tikzmarknode{alpha}{\highlight{blue}{$\alpha$}}}
    [ 
    {\tikzmarknode{r}{\highlight{Bittersweet}{$r_{t+1}$}}}  +  
    {\tikzmarknode{gamma}{\highlight{RoyalBlue}{$\gamma$}}}
    {\tikzmarknode{max}{\highlight{OliveGreen}{$\max\limits_a Q(s_{t+1},a)$}}} - 
    {\tikzmarknode{qt2}{\highlight{purple}{$Q(s_t, a_t)$}}} 
    ]
\end{equation*}

\begin{tikzpicture}[overlay,remember picture,>=stealth,nodes={align=left,inner ysep=1pt},<-]
    % For "Qt1"
    \path (qt.north) ++ (3.7,1.6em) node[anchor=south east,color=Mulberry!85] (ntext){\textsf{\footnotesize Q-value}};
    \draw [color=Mulberry](qt.north) |- ([xshift=0.8ex,color=Mulberry]ntext.south west);
    \path (qt2.north) ++ (-1.2,1.6em) node[anchor=south east,color=Mulberry!85] (qt2text){};
    \draw [color=Mulberry](qt2.north) |- ([xshift=-4.9ex,color=Mulberry]qt2text.south west);
    % For alpha
    \path (alpha.north) ++ (-0.2,-2.8em) node[anchor=south east,color=blue!85] (atext){\textsf{\footnotesize learning rate}};
    \draw [color=blue](alpha.south) |- ([xshift=-9.3ex,color=blue]atext.south east);
    % For r
    \path (r.north) ++ (-0.1,1.5em) node[anchor=north east,color=Bittersweet!85] (lijtext){\textsf{\footnotesize reward}};
    \draw [color=Bittersweet](r.north) |- ([xshift=-4.3ex,color=Bittersweet]lijtext.south east);
    %gamma
    \path (gamma.north) ++ (0.5,1.5em) node[anchor=north west,color=RoyalBlue!85] (gtext){\textsf{\footnotesize discount factor}};
    \draw [color=RoyalBlue](gamma.north) |- ([xshift=-2.9ex,color=RoyalBlue]gtext.south east);
    % For "l_i^max"
    \path (max.north) ++ (-1.2,-3.6em) node[anchor=south west,color=xkcdHunterGreen!85] (lmaxtext){\textsf{\scriptsize Maximum Q-Value in the next state}};
    \draw [color=xkcdHunterGreen](max.south) |- ([xshift=-5ex,color=xkcdHunterGreen]lmaxtext.north);
\end{tikzpicture}




\endinput

\hfil

\begin{minipage}{0.5\columnwidth}
\begin{equation*}
    \label{eq:ab_crypto}
    \hspace*{-6em}
    X_{i} = \frac{1}{\sum_{i=1}^{\tikzmarknode{n}{\highlight{purple}{N}}} 
    \sum_{j=1}^{\tikzmarknode{mi}{\highlight{blue}{$M_i$}}} 
    \tfrac{\tikzmarknode{lij}{\highlight{Bittersweet}{$l_i^j$}}}{\tikzmarknode{lmax}{\highlight{OliveGreen}{$l^{max}$}}}
    }
\end{equation*}
\vspace*{0.8\baselineskip}
\begin{tikzpicture}[overlay,remember picture,>=stealth,nodes={align=left,inner ysep=1pt},<-]
    % For "N"
    \path (n.north) ++ (0,1.8em) node[anchor=south east,color=Plum!85] (ntext){\textsf{\footnotesize number of objects}};
    \draw [color=Plum](n.north) |- ([xshift=-0.3ex,color=Plum]ntext.south west);
    % For "M_i"
    \path (mi.north) ++ (0,3.5em) node[anchor=north west,color=blue!85] (mitext){\textsf{\footnotesize number of other objects}};
    \draw [color=blue](mi.north) |- ([xshift=-0.3ex,color=blue]mitext.south east);
    % For "l_i^j"
    \path (lij.north) ++ (0,1.9em) node[anchor=north west,color=Bittersweet!85] (lijtext){\textsf{\footnotesize size of j\textsuperscript{th} service}};
    \draw [color=Bittersweet](lij.north) |- ([xshift=-0.3ex,color=Bittersweet]lijtext.south east);
    % For "l_i^max"
    \path (lmax.north) ++ (-2.7,-1.5em) node[anchor=north west,color=xkcdHunterGreen!85] (lmaxtext){\textsf{\footnotesize maximum obj size}};
    \draw [color=xkcdHunterGreen](lmax.south) |- ([xshift=-0.3ex,color=xkcdHunterGreen]lmaxtext.south west);
\end{tikzpicture}
\end{minipage}
\caption{Two Equations side-by-side using minipage and figure constructs.}

\endinput