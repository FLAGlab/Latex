% !TEX root = keeping-up.tex


%%
\section{Bibliography management}
\label{sec:biblio}

Bibliographies are manages by the bibtex package/specification. This will let us manage bibliographic references with ease, by defining them in a shorthand way, as in the \spy{local.bib} file, as shown in \fref{lst:bib} for the definition of a reference \cite{golendukhina22}.

\begin{lstlisting}[frame=lines, 
   label={lst:bib}, 
  caption={Bib entry example}]
@inproceedings{golendukhina22,
	author = {Valentina Golendukhina and Valentina Lenarduzzi and Michael Felderer},
	booktitle = {International Conference on AI Engineering - Software Engineering for AI},
	pages = {1--9},
	series = {CAIN'22},
	title = {What is Software Quality for AI Engineers? Towards a Thinning of the Fog},
	year = {2022}}
\end{lstlisting}

References are \textbf{NOT} text objects, therefore you should never do something like citing the code in \cite{islam20}. Instead you have to use the natbib package and the \spy{citet} command to display the authors' names as in \citet{islam20}.

Yo can play with the parameters of the biblatex package to programmatically modify bib entries and style.

\endinput